\documentclass{article}
\usepackage[utf8]{inputenc}
\usepackage{amssymb}
\usepackage{amsmath}

\title{Human Monitoring problem formulation}
\author{Catalina Obando Forero}
\date{January 2021}

\begin{document}

\maketitle

\section{General formulation}
Let  $\mathcal{X}_t =  \mathbb{R}^{C \times T} $ denote a temporal process of the brain activity of a human performing a task, measured by an EEG with $C$ channels and $T$ the total time of the recording. Given a task, a performance measurement $R_t \in \mathbb{R} $ is associated. \\

We want to monitor the human performance by finding a function $d: X_t \rightarrow{\mathbb{R}}$, such that $d(X_k) = R_t$. \\

\section{Example: Emergency Braking use case}
Consider a human driving a virtual car tasked with tightly follow a computer-controlled lead vehicle. This lead vehicle occasionally decelerates abruptly. The pilot is instructed to perform immediate emergency braking in these situations in order to avoid a crash.

The aim is to use pilot's brain activity measured by an EEG and behavioral response to:
\begin{enumerate}
    \item Discriminate an emergency situation to a normal driving situation
    \item Assist the pilot in the emergency situation by detecting the emergency before the behavioral break action occurs 
    \item Define a coherent performance $R_t$ of the pilot's action in emergency situations
    \item Predict pilots performance form neurophysiological responses
\end{enumerate}

\subsection{Discriminate an emergency and normal driving}
This is the easy case: classify braking events to no braking event -> obtained high accuracy with different ML methods (use to benchmark) 

\subsection{Assist the pilot in an emergency situation}
Show how much time is save by detecting the emergency from the eeg, show how many collisions could be avoided, ... 

\subsection{Performance}
Performance defined as collision or not collision.
Performance defined as a reaction time.

\subsection{Predict pilots performance}



\section{Methods}

\subsection{DeepNet}
Let
$f: X_k \rightarrow{\mathcal{Y}} $
stand for a predictive model that belongs to a parametric set denoted $\mathcal{F}$, where  $\mathcal{X}_k = \left( \mathbb{R}^{C \times T} \right)^{(2k+1)} $ with $C$ the number of EEG channels and $T$ the second of a EEG segment.\\

$f$ takes as input an ordered sequence of $2k+1$ neighboring segments of signal and outputs a probability vector $p\in \mathcal{Y}$.\\

The machine learning problem tackled then reads

$\hat{f}  = \arg \min \mathbb{E}_{x,y\in \mathcal{X}_k \times \mathcal{Y}} \left[ \ell (f(x),y) \right]$\\


whenever $k>0$ the neural network has access to the temporal context of the segment of signal to classify, it is the \cal{temporal stage classification problem}, and when $k=0$ the problem boils down to the standard formulation of stage classification. 

\subsection{KL}



%%%%%%%%%%%%%%
\end{document}


